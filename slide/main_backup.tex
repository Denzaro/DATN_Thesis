% set documentclass to widescree 16:9 ratio
\documentclass[aspectratio=169]{beamer}

% Load UofC_dynamic style file
\usepackage{UofC_dynamic}

% Title and author data
\title{Type to add title of the presentation/talk} %Title
\subtitle{Type to add subtitle of the presentation/talk} %Subtitle
% Had to use \texorpdfstring{//} to ensure new line in presenter details
\author[Presenter's Name]{Presenter's Name \texorpdfstring{\\} % presenter info
        Presenter's title / additional designations \texorpdfstring{\\}
        Faculty of / Department of / additional designations}
\date{\today} % date

\begin{document}

% ---------------------------------------------------------
% 1. Title Page
% Uses the Title Background, no footline/headline
% ---------------------------------------------------------
\CustomTitlePage

% ---------------------------------------------------------
% 2. Type A: Full Page Content Frame
% Uses Content Background + Footline
% Usage \begin{frame} <content> /end{frame}
% ---------------------------------------------------------
\begin{frame}{Standard Content Frame}
    \begin{itemize}
        \item This is a standard full-page content frame.
        \item It uses the Times New Roman font.
        \item It has the content background image.
        \item It includes the footline defined in the style file.
    \end{itemize}

    \[ E = mc^2 \]
\end{frame}

% ---------------------------------------------------------
% 3. Type B: Two Column content frame
% Usage:   \begin{SplitFrame} 
         %        <Left Side content> 
         % \nextcolumn 
         %        <Right Side content> 
         % \end{SplitFrame}
% ---------------------------------------------------------
\begin{SplitFrame}{Comparison Analysis}
    \textbf{Left Side}
    
    Here is some text on the left side of the screen.
    \begin{itemize}
        \item Point A
        \begin{itemize}
            \item Subpoint A.a
            \begin{itemize}
                \item Subsubpoint A.a.a
        \end{itemize}
        \end{itemize}
        \item Point B
    \end{itemize}
    
    \nextcolumn % This command inserts the orange line and switches columns
    
    \textbf{Right Side}
    
    Here is the text on the right side. The orange line separates these two columns perfectly.
    \begin{enumerate}
        \item Point 1
        \item Point 2
        \item Point 3
    \end{enumerate}
\end{SplitFrame}

% ---------------------------------------------------------
% 4. Type E: Special Heading Only Frame
% Largest possible heading, no other content
% Usage: \BigHeadingFrame<heading>
% ---------------------------------------------------------
\BigHeadingFrame{This slide is for one big, bold statement. Bullet points can't compete!}

% ---------------------------------------------------------
% 5. Type C: Full Page Image Frame
% No headline. Replaces background locally.
% ---------------------------------------------------------
% Ensure you have an image named 'image4.jpg' or change this
% Usage: \FullImageFrame<image_path><width><caption><label>

\FullImageFrame{
    assets/image4.jpg
}{
    0.7\textwidth % <--- Image Width (70% of the text width)
}{
    This is a detailed figure illustrating the main findings of the study.
}
{
    fig:myFig
}

% ---------------------------------------------------------
% 6. Type D: Thank You Page (Constraint 3d)
% Uses Title Background
% ---------------------------------------------------------
% Usage: \ThankYouFrame<text><more_info><contact_details>

\ThankYouFrame{
    Thank you for attending!\\
    and/or other concluding message
}{
    For more information go to \url{ucalgary.ca/webaddress}
}{
    Presenter's name\\
    presentersemail@ucalgary.ca\\
    Phone number / Twitter handle / additional contact info
}

\end{document}