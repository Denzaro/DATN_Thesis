% set documentclass to widescree 16:9 ratio
\documentclass[aspectratio=169]{beamer}

% Load UofC_dynamic style file
\usepackage{UofC_dynamic}
\usepackage{booktabs} % For better tables
\usepackage[utf8]{inputenc}
\usepackage[vietnamese]{babel}

% Title and author data
\title{Xây dựng hệ thống mô phỏng dự báo và điều chỉnh đèn giao thông dựa trên dữ liệu snapshot từ camera giao thông sử dụng học sâu và mô phỏng SUMO} %Title
% Had to use \texorpdfstring{//} to ensure new line in presenter details
\author[N.T. Nghĩa \& N.N. Đăng]{Ngô Trọng Nghĩa (21161155) - Nguyễn Nhựt Đăng (21119062) \texorpdfstring{\\} % presenter info
        Khoa Điện - Điện Tử \texorpdfstring{\\}
        Trường Đại Học Sư Phạm Kỹ Thuật TP.HCM}
\date{\today} % date

\begin{document}

% ---------------------------------------------------------
% 1. Title Page
% ---------------------------------------------------------
\CustomTitlePage

% ---------------------------------------------------------
% 2. Outline
% ---------------------------------------------------------
\begin{frame}{Nội Dung Báo Cáo}
    \begin{itemize}
        \item \textbf{I. GIỚI THIỆU}: TỔNG QUAN DỰ ÁN
        \item \textbf{II. THIẾT KẾ HỆ THỐNG}: KIẾN TRÚC VÀ CÁC MÔ-ĐUN CHÍNH
        \item \textbf{III. KẾT QUẢ}: KẾT QUẢ THỰC NGHIỆM VÀ SO SÁNH KỊCH BẢN
        \item \textbf{IV. ĐÁNH GIÁ}: ĐÁNH GIÁ MÔ HÌNH
        \item \textbf{IV. KẾT LUẬN}: TỔNG KẾT VÀ HƯỚNG PHÁT TRIỂN
    \end{itemize}
\end{frame}

% ---------------------------------------------------------
% I. INTRODUCTION
% ---------------------------------------------------------
\BigHeadingFrame{I. GIỚI THIỆU}

\begin{SplitFrame}{Lý Do Chọn Đề Tài}
    \textbf{Thách thức đô thị}
    \begin{itemize}
        \item Đô thị hóa nhanh dẫn đến tắc nghẽn nghiêm trọng.
        \item Thiệt hại kinh tế lớn và ô nhiễm môi trường.
        \item Đèn giao thông chu kỳ cố định không hiệu quả.
    \end{itemize}
    
    \nextcolumn 
    
    \textbf{Giải pháp đề xuất}
    \begin{itemize}
        \item Ứng dụng **Deep Learning** (YOLO, LSTM) để giám sát thời gian thực.
        \item Sử dụng **Mô phỏng SUMO** để thử nghiệm an toàn.
        \item **Điều khiển thích nghi** (Adaptive Control) để tối ưu dòng xe.
    \end{itemize}
\end{SplitFrame}

\begin{frame}{Mục Tiêu Đề Tài}
    \begin{itemize}
        \item \textbf{Thu thập dữ liệu}: Tự động tải ảnh snapshot từ camera giao thông công cộng.
        \item \textbf{Xử lý}: Tăng cường chất lượng ảnh (Real-ESRGAN) và phát hiện xe (YOLOv11).
        \item \textbf{Dự báo}: Dự báo mức độ tắc nghẽn (Gridlock Levels) sử dụng LSTM.
        \item \textbf{Điều khiển}: Điều khiển đèn giao thông thích nghi dùng Học Tăng Cường (Reinforcement Learning - DurationAgent).
        \item \textbf{Đánh giá}: Kiểm chứng hiệu quả trên môi trường mô phỏng SUMO.
    \end{itemize}
\end{frame}

% ---------------------------------------------------------
% II. SYSTEM DESIGN
% ---------------------------------------------------------
\BigHeadingFrame{II. THIẾT KẾ HỆ THỐNG}

\FullImageFrame{
    ../fig/system_block_diagram.png
}{
    0.7\textwidth
}{
    Sơ đồ khối hệ thống: Từ thu thập dữ liệu đến điều khiển mô phỏng
}
{
    fig:system_block
}

\begin{frame}{1. Thu Thập & Tiền Xử Lý Dữ Liệu}
    \begin{columns}
        \begin{column}{0.5\textwidth}
            \textbf{Thu thập dữ liệu}
            \begin{itemize}
                \item Cào (Scrape) ảnh snapshot từ web camera.
                \item Lưu trữ Google Drive (Phân cấp: Địa điểm/Ngày/Giờ).
            \end{itemize}
            \vspace{0.5cm}
            \textbf{Tiền xử lý}
            \begin{itemize}
                \item **Real-ESRGAN**: Nâng cao độ phân giải ảnh thấp.
                \item Chia nhỏ ảnh thành 4 phần (quadrants) để xử lý chi tiết.
            \end{itemize}
        \end{column}
        \begin{column}{0.5\textwidth}
            \includegraphics[width=\textwidth]{../fig/upscale_image.png}
        \end{column}
    \end{columns}
\end{frame}

\begin{frame}{2. Nhận Diện Phương Tiện (YOLOv11)}
    \begin{columns}
        \begin{column}{0.55\textwidth}
            \includegraphics[width=\textwidth]{../fig/vehicle_detector.png}
        \end{column}
        \begin{column}{0.45\textwidth}
            \textbf{Phương pháp Lai (Hybrid)}
            \begin{itemize}
                \item **Nhận diện Ô tô**: Dùng YOLOv11n (Bounding Boxes).
                \item **Ước lượng Xe máy**:
                \begin{itemize}
                    \item Khó nhận diện từng xe khi mật độ cao.
                    \item Dùng **Độ phủ vùng quan tâm (ROI Coverage \%)** + Phân đoạn.
                    \item Quy đổi \% diện tích $\to$ Số lượng xe.
                \end{itemize}
            \end{itemize}
        \end{column}
    \end{columns}
\end{frame}

\begin{frame}{3. Dự Báo Tắc Nghẽn (LSTM)}
    \begin{columns}
        \begin{column}{0.48\textwidth}
            \textbf{Kiến trúc Mô hình}
            \begin{itemize}
                \item \textbf{Đầu vào}: Dữ liệu giao thông 10 phút trước.
                \item \textbf{LSTM Layers}: Học phụ thuộc thời gian.
                \item \textbf{Đầu ra}: Mức độ tắc nghẽn (0-5).
            \end{itemize}
        \end{column}
        \begin{column}{0.48\textwidth}
            \centering
            \includegraphics[width=\textwidth,height=0.65\textheight,keepaspectratio]{../fig/fig_3_7_lstm_architecture.png}
        \end{column}
    \end{columns}
\end{frame}

\begin{frame}{4. Mô Phỏng và Điều Khiển (SUMO)}
    \begin{columns}[T]
        \begin{column}{0.35\textwidth}
            \begin{itemize}
                \item \textbf{Môi trường}: SUMO.
                \item \textbf{Bản đồ}: OpenStreetMap.
                \item \textbf{Agent: DurationAgent}
                \begin{itemize}
                    \item Actor-Critic RL.
                    \item Điều chỉnh đèn xanh (40-80s).
                    \item Tối đa Thông lượng.
                \end{itemize}
            \end{itemize}
        \end{column}
        \begin{column}{0.62\textwidth}
            \centering
            \includegraphics[width=\textwidth,height=0.7\textheight,keepaspectratio]{../fig/fig_II4_sumo_control.png}
        \end{column}
    \end{columns}
\end{frame}

% ---------------------------------------------------------
% III. RESULTS
% ---------------------------------------------------------
\BigHeadingFrame{III. KẾT QUẢ}

\begin{frame}{1. Kết Quả Nhận Diện}
    \begin{columns}
        \begin{column}{0.5\textwidth}
            \centering
            \includegraphics[width=0.9\textwidth]{../fig/detect_q1_tl.png} \\
            \small{Độ phủ ROI cho Xe máy}
        \end{column}
        \begin{column}{0.5\textwidth}
            \centering
            \includegraphics[width=0.9\textwidth]{../fig/result_segment.jpg} \\
            \small{Kết quả nhận diện tích hợp cuối cùng}
        \end{column}
    \end{columns}
\end{frame}

\begin{frame}{2. Dashboard và Trực Quan Hóa}
    \centering
    \includegraphics[width=0.85\textwidth]{../fig/gridlock_prediction_dashboard.png}
    \begin{itemize}
        \item Giám sát thời gian thực dòng phương tiện và mức độ tắc nghẽn.
        \item So sánh giữa Dự báo (LSTM) và Thực tế.
    \end{itemize}
\end{frame}

\begin{frame}{3. Hiệu Quả Điều Khiển: Thông Lượng}
    \centering
    \includegraphics[width=0.85\textwidth]{../fig/results_throughput_comparison.png}
    \vspace{0.2cm}
    \begin{itemize}
        \item **DurationAgent (Xanh)** vượt trội so với Baseline (Cam).
        \item **Thông Lượng (Throughput)**: Tăng \textbf{+88.5\%} (115.9 $\to$ 218.6 xe).
    \end{itemize}
\end{frame}

\begin{SplitFrame}{Hiệu Quả Điều Khiển: Hàng Đợi và Reward}
    \textbf{So Sánh Độ Dài Hàng Đợi}
    \includegraphics[width=\textwidth]{../fig/results_queue_length_comparison.png}
    \begin{itemize}
        \item Hàng đợi tăng nhẹ, nhưng dòng xe cải thiện lớn.
        \item Đánh đổi để đạt thông lượng cao hơn.
    \end{itemize}
    
    \nextcolumn 
    
    \textbf{Phần Thưởng Tích Lũy}
    \includegraphics[width=\textwidth]{../fig/results_reward_comparison.png}
    \begin{itemize}
        \item **Cải thiện Reward**: \textbf{+40.5\%}.
        \item Hệ thống tối ưu hóa hiệu quả di chuyển tổng thể.
    \end{itemize}
\end{SplitFrame}

\begin{frame}{5. So Sánh Tổng Thể}
    \centering
    \begin{table}
        \begin{tabular}{lccc}
            \toprule
            \textbf{Chỉ Số} & \textbf{Baseline (Cố định)} & \textbf{DurationAgent} & \textbf{Cải thiện} \\
            \midrule
            Thông lượng (xe) & 115.9 & 218.6 & \textbf{+88.5\%} \\
            Reward & 12.2 & 17.1 & \textbf{+40.5\%} \\
            Hàng đợi (xe) & 2.25 & 4.25 & N/A (Đánh đổi) \\
            Độ ổn định Hàng đợi & 2.50 & 4.58 & +83.7\% \\
            \bottomrule
        \end{tabular}
    \end{table}
    \vspace{0.5cm}
    \textbf{Kết luận}: AI Agent học thành công cách "xả" xe trong giờ cao điểm, tăng gấp đôi năng lực mạng lưới so với định thời cố định.
\end{frame}

% ---------------------------------------------------------
% IV. CONCLUSION
% ---------------------------------------------------------
\BigHeadingFrame{IV. KẾT LUẬN}

\begin{frame}{Tổng Kết và Hướng Phát Triển}
    \textbf{Tổng Kết}
    \begin{itemize}
        \item Xây dựng thành công quy trình khép kín: Camera $\to$ Dữ liệu $\to$ AI $\to$ Mô phỏng.
        \item Phương pháp **YOLOv11 + ROI** giải quyết bài toán đếm xe máy tại Việt Nam.
        \item **LSTM** cảnh báo sớm xu hướng tắc nghẽn.
        \item **Agent RL** tăng thông lượng gần \textbf{90\%}.
    \end{itemize}
    
    \vspace{0.5cm}
    
    \textbf{Hướng Phát Triển}
    \begin{itemize}
        \item Triển khai trên thiết bị Edge (Jetson Nano/Orin).
        \item Tích hợp với bộ điều khiển đèn thực tế.
        \item Mở rộng quy mô mạng lưới nhiều nút giao hơn.
    \end{itemize}
\end{frame}


% ---------------------------------------------------------
% THANK YOU
% ---------------------------------------------------------
\ThankYouFrame{
    Cảm Ơn Quý Thầy Cô và Các Bạn Đã Lắng Nghe!
}{
    Source Code và Demo: \url{github.com/YourRepo}
}{
    \textbf{Nguyễn Nhựt Đăng} \\
    dang.nguyen@student.hcmute.edu.vn \\
    \textbf{Ngô Trọng Nghĩa} \\
    nghia.ngo@student.hcmute.edu.vn
}

\end{document}