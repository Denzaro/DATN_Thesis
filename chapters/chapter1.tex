\chapter{TỔNG QUAN}

\section{GIỚI THIỆU}

Trong bối cảnh đô thị hóa nhanh chóng và sự gia tăng mật độ phương tiện giao thông, việc quản lý và điều tiết giao thông hiệu quả đã trở thành một trong những thách thức lớn nhất mà các thành phố hiện đại phải đối mặt. Theo báo cáo của Tổ chức Hợp tác và Phát triển Kinh tế, tắc nghẽn giao thông không chỉ gây ra thiệt hại kinh tế hàng tỷ USD mỗi năm mà còn là nguyên nhân chính dẫn đến ô nhiễm không khí, tiêu thụ nhiên liệu không hiệu quả và giảm chất lượng cuộc sống của người dân~\cite{OECD2019Traffic}.

Sự phát triển của công nghệ thông tin và truyền thông cùng với sự xuất hiện của các hệ thống giao thông thông minh đã mở ra những cơ hội mới trong việc giải quyết các vấn đề giao thông. Đặc biệt, việc ứng dụng các kỹ thuật học máy và học sâu (Deep Learning) vào phân tích dữ liệu giao thông đã cho thấy những kết quả đầy hứa hẹn trong việc dự báo và tối ưu hóa luồng giao thông~\cite{Zhang2018DeepLearning}.

Hệ thống camera giao thông hiện đại có khả năng thu thập một lượng lớn dữ liệu hình ảnh theo thời gian thực, tạo ra những "snapshot" phản ánh tình trạng giao thông tại các điểm quan sát. Những dữ liệu này, khi được xử lý bằng các thuật toán học sâu tiên tiến, có thể cung cấp thông tin quý giá về mật độ phương tiện, tốc độ di chuyển, và các mẫu hành vi giao thông, từ đó làm cơ sở cho việc dự báo và điều tiết giao thông hiệu quả.

\section{MỤC TIÊU}

Mục tiêu của đề tài là xây dựng một hệ thống tích hợp giữa thu thập dữ liệu ảnh snapshot từ camera giao thông, xử lý bằng các phương pháp học sâu, và mô phỏng trên nền tảng SUMO nhằm dự báo lưu lượng phương tiện và hỗ trợ điều chỉnh tín hiệu đèn giao thông một cách thích ứng. Hệ thống hướng tới việc khai thác dữ liệu hình ảnh để cung cấp thông tin định lượng về tình trạng giao thông, từ đó nâng cao hiệu quả điều khiển nút giao trong bối cảnh giao thông đô thị.

Cụ thể, đề tài tập trung vào các mục tiêu sau:
\begin{itemize}
    \item Xây dựng quy trình thu thập và xử lý ảnh snapshot từ camera giao thông, chuyển đổi dữ liệu hình ảnh thô thành các thông số giao thông có cấu trúc như lưu lượng xe, mật độ phương tiện và phân bố phương tiện theo làn, thông qua các mô hình học sâu.

    \item Tích hợp các thông số giao thông trích xuất được vào môi trường mô phỏng SUMO nhằm tái hiện trạng thái giao thông tại nút giao hoặc khu vực nghiên cứu, phục vụ cho việc phân tích và đánh giá hoạt động của hệ thống tín hiệu đèn.

    \item Phát triển mô hình dự báo ngắn hạn lưu lượng phương tiện dựa trên chuỗi dữ liệu thời gian thu được từ camera và kết quả mô phỏng, với mục tiêu ước lượng xu hướng biến động giao thông trong các khoảng thời gian kế tiếp.

    \item Thiết kế và đánh giá chiến lược điều chỉnh tín hiệu đèn giao thông dựa só lượng phương tiện hiện tại, nhằm tối ưu hóa các chỉ tiêu vận hành như thời gian chờ trung bình tại nút giao, số lần dừng xe và mức độ ùn tắc.

    \item Đánh giá hiệu quả của hệ thống thông qua các thước đo như sai số dự báo lưu lượng, thời gian chờ trung bình, chiều dài hàng đợi và khả năng áp dụng trong thực tế, từ đó đề xuất hướng cải tiến và mở rộng cho các hệ thống điều khiển tín hiệu giao thông thông minh.
\end{itemize}

Tóm lại, mục tiêu của đề tài là xây dựng một công cụ hỗ trợ thông minh cho công tác quản lý và điều khiển tín hiệu giao thông, kết hợp giữa các phương pháp học sâu và mô phỏng giao thông. Hệ thống cho phép chuyển hóa dữ liệu hình ảnh snapshot từ camera giao thông thành các thông tin định lượng về lưu lượng phương tiện, phục vụ cho bài toán dự báo và điều chỉnh tín hiệu đèn. Qua đó, đề tài hướng tới việc cải thiện hiệu quả vận hành tại các nút giao, giảm thời gian chờ và mức độ ùn tắc, góp phần nâng cao chất lượng dịch vụ giao thông đô thị.

\section{PHƯƠNG PHÁP NGHIÊN CỨU}

Để đạt được các mục tiêu đã đề ra, Nghiên cứu này áp dụng một quy trình nghiên cứu khoa học chặt chẽ, kết hợp giữa nghiên cứu lý thuyết, thực nghiệm mô phỏng và phân tích đánh giá định lượng. Các phương pháp cụ thể được triển khai như sau:

Phương pháp nghiên cứu lý thuyết:

Trước hết, đề tài tiến hành khảo sát, tổng hợp và phân tích các cơ sở lý thuyết và công trình nghiên cứu liên quan đến bài toán giao thông đô thị và các phương pháp tiếp cận hiện đại trong lĩnh vực thị giác máy tính, học sâu và mô phỏng giao thông. Nội dung nghiên cứu tập trung vào các khái niệm nền tảng về hệ thống giao thông, đặc trưng dòng xe, cũng như các chỉ số thường được sử dụng để đánh giá trạng thái giao thông như lưu lượng, mật độ và thời gian chờ.

Tiếp theo, đề tài nghiên cứu các mô hình học sâu phục vụ nhận dạng và đếm phương tiện từ dữ liệu hình ảnh, trong đó tập trung vào các mô hình phát hiện đối tượng một giai đoạn như YOLO, đặc biệt là YOLOv11, đã được chứng minh hiệu quả trong các bài toán đếm xe và ước lượng lưu lượng giao thông từ camera giám sát. Song song với đó, các mô hình dự báo chuỗi thời gian, tiêu biểu là Long Short-Term Memory (LSTM), được khảo sát nhằm khai thác mối quan hệ theo thời gian của dữ liệu lưu lượng, phục vụ cho bài toán dự báo ngắn hạn.

Bên cạnh đó, đề tài tổng hợp các nghiên cứu liên quan đến mô phỏng giao thông vi mô bằng SUMO, cũng như các hướng tiếp cận tích hợp giữa dữ liệu thu thập từ camera, kết quả nhận dạng phương tiện và mô hình mô phỏng. Các công trình về điều khiển tín hiệu giao thông dựa trên dữ liệu mô phỏng và dự báo cũng được phân tích nhằm làm rõ cách thức ánh xạ từ dữ liệu quan sát sang các tham số điều khiển trong môi trường mô phỏng. Từ những cơ sở lý thuyết này, đề tài xây dựng nền tảng cho việc lựa chọn kiến trúc hệ thống, xác định các tham số mô hình và đề xuất các tiêu chí đánh giá hiệu quả như độ chính xác dự báo, lưu lượng, mật độ phương tiện và thời gian chờ tại nút giao.

Phương pháp thực nghiệm mô phỏng:
\begin{itemize}
    \item Thu thập dữ liệu snapshot camera và xử lý bằng học sâu: Quá trình được khởi đầu bằng việc tự động hóa thu thập dữ liệu snapshot từ các website camera giám sát thông qua các kỹ thuật scraping, kết hợp với Google API để truyền tải và lưu trữ dữ liệu tập trung trên Google Drive, đảm bảo tính liên tục và tính sẵn sàng cao cho việc phân tích quy mô lớn. Ngay sau đó, sức mạnh của mô hình học sâu tiên tiến YOLOv11 được khai thác để nhận diện và đếm chính xác các loại phương tiện trong từng khung hình, từ đó bóc tách các thông số then chốt như lưu lượng xe và mật độ giao thông theo từng mốc thời gian và vị trí cụ thể. 
    
    \item Xây dựng mô hình dự báo mật độ giao thông: None
    
    \item Xây dựng mô hình điều chỉnh tín hiệu đèn giao thông: None
    
    \item Tích hợp mô phỏng giao thông với SUMO: None
    
    \item Tích hợp module điều khiển giao thông: None
    
    \item Xây dựng dashboard trực quan: Triển khai giao diện hiển thị gồm luồng xe chạy như mô phỏng (trong SUMO), biểu đồ mật độ theo thời gian, kết quả dự báo LSTM, và so sánh hiệu quả giữa hai kịch bản (có/không điều tiết). Dashboard sẽ giúp trực quan hóa kết quả và hỗ trợ phân tích.
    
    \item Rút ra kết luận, đề xuất cải tiến và kiến nghị ứng dụng thực tiễn: từ kết quả thực nghiệm và mô phỏng, chỉ rõ giới hạn của nghiên cứu, gợi ý mở rộng (ví dụ mở rộng mạng lưới, loại phương tiện đa dạng, tích hợp dữ liệu thời tiết, sự cố…).
\end{itemize}

Phương pháp phân tích đánh giá:
\begin{itemize}
    \item \textbf{Đánh giá độ chính xác của khối phát hiện và đếm phương tiện từ camera: } Kết quả đếm phương tiện tự động được đối chiếu với số liệu đếm thủ công hoặc các nguồn dữ liệu tham chiếu (nếu có), nhằm xác định mức độ chính xác và độ tin cậy của mô hình trong các điều kiện giao thông khác nhau.
    
    \item \textbf{Phân tích hiệu năng của mô hình LSTM: } None
    
    \item \textbf{Phân tích kết quả mô phỏng giao thông bằng SUMO: } None
    
    \item \textbf{Đánh giá tổng thể hệ thống: } None.

    \item \textbf{Trực quan hóa và phân tích kết quả trên dashboard: } Trình bày và so sánh các biểu đồ, luồng giao thông và chỉ số vận hành trước và sau điều chỉnh đèn tín hiệu, từ đó rút ra nhận xét về hiệu quả điều tiết và lợi ích mang lại cho công tác quản lý giao thông đô thị.
\end{itemize}

\section{GIỚI HẠN NGHIÊN CỨU}

Mặc dù hệ thống nghiên cứu đã nỗ lực tích hợp các thành phần thu thập dữ liệu hình ảnh, học sâu, dự báo lưu lượng và mô phỏng điều tiết giao thông thành một quy trình tương đối hoàn chỉnh, song vẫn còn tồn tại một số hạn chế cần được xem xét. Trước hết, việc sử dụng ảnh snapshot từ camera giao thông chỉ phản ánh trạng thái giao thông tại những thời điểm rời rạc, mang tính “tĩnh”, và chịu ảnh hưởng đáng kể từ các điều kiện quan sát như góc đặt camera, điều kiện chiếu sáng, mức độ che khuất và chất lượng hình ảnh. Do đó, hiệu quả phát hiện và đếm phương tiện của mô hình YOLOv11 có thể suy giảm trong các điều kiện môi trường bất lợi (mưa, sương mù, bóng đổ) hoặc tại các khu vực giao thông phức tạp, nơi mật độ phương tiện cao và làn đường không được phân định rõ ràng.

Bên cạnh đó, dữ liệu thu thập tại một hoặc một số vị trí giao thông cụ thể chưa đủ để phản ánh đầy đủ đặc trưng của toàn bộ mạng lưới giao thông đô thị cũng như sự đa dạng của các điều kiện vận hành theo thời gian (giờ cao điểm, ngày lễ, hoặc các tình huống sự cố). Điều này có thể làm hạn chế khả năng khái quát hóa kết quả dự báo và mô phỏng sang các bối cảnh giao thông khác.

Ngoài ra, do hạn chế về kinh phí, nghiên cứu chỉ có thể khai thác nguồn dữ liệu ảnh snapshot từ các camera giao thông công khai, vốn có vị trí lắp đặt cố định và không thể điều chỉnh góc quay. Hệ quả là nhiều ảnh thu được có góc nhìn không thuận lợi (góc quay hẹp, góc xiên), bị che khuất bởi các vật cản như cây cối, biển báo hoặc cột đèn, cũng như có chất lượng hình ảnh không đồng đều. Những yếu tố này làm giảm độ chính xác của quá trình nhận diện và đếm phương tiện, từ đó dẫn đến sai số trong việc ước lượng mật độ giao thông và ảnh hưởng đến các bước phân tích tiếp theo của hệ thống.

Hệ thống cũng không thể lấy được toàn bộ khung giờ trong ngày để phân tích do chỉ tập trung vào các giờ cao điểm buổi sáng và chiều. Với những khung giờ khác như chiều muộn hoặc ban đêm, chất lượng hình ảnh thấp hơn do điều kiện ánh sáng kém và trong một vài trường hợp thời tiết xấu như mưa thì sẽ ảnh hưởng đến khả năng nhận diện và đếm phương tiện của hệ thống.

Một thách thức lớn khác nằm ở tính biến động của hạ tầng camera. Việc thay đổi vị trí hay góc quan sát khiến hệ thống nhận diện và đếm phương tiện phải tái hiệu chỉnh. Điều này không chỉ gây tốn kém về nguồn lực mà còn là rào cản trong việc duy trì độ chính xác và tính ổn định của dữ liệu theo thời gian.

Về phần mô hình học sâu và dự báo bằng Long Short-Term Memory (LSTM), mặc dù có khả năng dự báo mật độ giao thông cho 15 phút tiếp theo, nhưng việc huấn luyện chỉ trên dữ liệu từ các mốc thời gian cố định và điểm giao thông nhất định khiến mô hình có thể kém chính xác khi đối mặt với tình huống bất thường (như tai nạn, thay đổi bất ngờ lưu lượng, thời tiết cực đoan) mà không có dữ liệu học trước. Hơn nữa, mô hình dự báo chỉ tập trung vào một biến chính - mật độ phương tiện - và chưa tích hợp đầy đủ các yếu tố khác như loại phương tiện, tốc độ, hành vi người lái, ảnh hưởng từ tín hiệu đèn hoặc thay đổi hành lang giao thông.

Phần mô phỏng giao thông với SUMO dù được “khớp” với dữ liệu thực tế đến mức có thể nhưng vẫn có giới hạn vì mô phỏng luôn là bản sao không hoàn hảo của môi trường thực. Mạng lưới mô phỏng có thể chưa mô hình hóa đầy đủ mọi chiều của thực tế như sự tương tác phức tạp giữa các phương tiện, hành vi bất định, ảnh hưởng thời tiết, người đi bộ, xe máy nhỏ, hoặc việc vi phạm giao thông - những yếu tố rất phổ biến tại đô thị như Hồ Chí Minh. Việc điều tiết giao thông thông qua thuật toán cũng đặt giả định rằng các thông số mô hình và dữ liệu đầu vào là đồng nhất và ổn định, trong khi thực tế có thể thay đổi nhanh và không thể đo trước hết.

Cuối cùng, việc xây dựng dashboard trực quan để hiển thị kết quả mô phỏng, dự báo và điều tiết cũng gặp giới hạn do khả năng phản ánh toàn bộ thực tế - dashboard chỉ hiển thị dữ liệu và mô phỏng ở mức độ “có thể” và phù hợp với giả định nghiên cứu. Những quyết định điều tiết đưa ra từ mô hình có thể chưa tính tới đầy đủ chi phí thực thi, điều kiện vận hành thực tế, phản ứng của người tham giao thông hoặc các yếu tố tổ chức giao thông ngoài mô hình.

Tóm lại, các giới hạn nghiên cứu chính bao gồm: dữ liệu nguồn (chỉ snapshot camera, tại các điểm giới hạn), khả năng khái quát hóa kết quả mô hình và mô phỏng, độ chính xác và phạm vi của mô hình dự báo, sự đơn giản hóa môi trường mô phỏng và giả định điều tiết, và khả năng ứng dụng thực tế bị chi phối bởi nhiều yếu tố ngoài mô hình. Hiểu và thừa nhận các giới hạn này giúp bạn đọc đánh giá đúng mức độ đóng góp của nghiên cứu, và tạo nền tảng cho các nghiên cứu tiếp theo.

\section{BỐ CỤC}

\textbf{Chương 1: Tổng quan} - Trình bày bối cảnh, động lực nghiên cứu, mục tiêu và phương pháp nghiên cứu.

\textbf{Chương 2: Cơ sở lý thuyết} - Tổng quan các kiến thức nền tảng về xử lý ảnh, học sâu, dự báo chuỗi thời gian và học tăng cường.

\textbf{Chương 3: Thiết kế hệ thống} - Trình bày kiến trúc tổng thể của hệ thống và thiết kế chi tiết các mô-đun.

\textbf{Chương 4: Kết quả và thảo luận} - Mô tả quá trình triển khai hệ thống và các thí nghiệm đánh giá.

\textbf{Chương 5: Kết luận và hướng phát triển} - Tổng kết những đóng góp của nghiên cứu và đề xuất hướng phát triển tương lai.