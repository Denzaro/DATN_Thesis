\chapter{KẾT LUẬN VÀ HƯỚNG PHÁT TRIỂN}

\section{KẾT LUẬN}

Sau quá trình nghiên cứu và thực hiện đề tài "Xây dựng hệ thống mô phỏng dự báo và điều chỉnh đèn giao thông dựa trên dữ liệu snapshot từ camera giao thông sử dụng học sâu và mô phỏng SUMO", nhóm thực hiện đã hoàn thành các mục tiêu đề ra và đạt được những kết quả khả quan. DurationAgent sử dụng LSTM và Actor-Critic training đã chứng minh hiệu quả vượt trội trong việc tối ưu hóa thông lượng mạng lưới.

Các kết quả chính đạt được bao gồm:

\begin{enumerate}
    \item \textbf{Hiệu quả điều khiển vượt trội:} Hệ thống đã chứng minh khả năng cải thiện thông lượng mạng lưới lên tới \textbf{+88.5\%} và tăng điểm thưởng tổng thể (Reward) thêm \textbf{+40.5\%} so với phương pháp điều khiển cố định (Baseline). Mặc dù độ dài hàng đợi trung bình tăng nhẹ, đây là sự đánh đổi chiến lược hợp lý để giải quyết lưu lượng giao thông lớn.
    
    \item \textbf{Xây dựng quy trình xử lý dữ liệu toàn diện:} Nhóm đã giải quyết hiệu quả thách thức từ dữ liệu ảnh snapshot chất lượng thấp thông qua các kỹ thuật tiền xử lý và mô hình tăng cường độ phân giải (Real-ESRGAN), kết hợp với mô hình YOLOv11 nhận diện phương tiện chính xác.
    
    \item \textbf{Dự báo tắc nghẽn chính xác:} Mô hình LSTM đã thể hiện khả năng nắm bắt tốt các đặc trưng chuỗi thời gian không gian của lưu lượng giao thông, cung cấp thông tin cảnh báo sớm quan trọng cho hệ thống điều khiển.
    
    \item \textbf{Khung mô phỏng khép kín (Closed-loop):} Hệ thống đã kết nối thành công luồng dữ liệu thời gian thực từ mô hình dự báo sang phần mềm mô phỏng SUMO thông qua giao thức TraCI, tạo tiền đề cho việc triển khai thực tế.
\end{enumerate}

\section{HẠN CHẾ CỦA ĐỀ TÀI}

Mặc dù đã đạt được những kết quả tích cực, hệ thống vẫn tồn tại một số hạn chế cần được xem xét:

\begin{itemize}
    \item \textbf{Thích ứng với sự cố bất ngờ (Sudden Incidents):} Agent hiện tại cần khoảng 30-60 giây để thích ứng sau khi có sự cố giao thông (như tai nạn hay đóng làn).
    \item \textbf{Phối hợp đa giao lộ (Multi-TLS Coordination):} Các agent hiện hoạt động độc lập tại từng giao lộ, chưa có cơ chế giao tiếp và phối hợp tối ưu trên toàn mạng lưới (Network-wide coordination).
    \item \textbf{Xử lý tình trạng kẹt cứng (Extreme Congestion):} Khi lưu lượng vượt quá khả năng thông hành của đường (bão hòa), hiệu quả điều chỉnh tín hiệu đèn bị giới hạn nếu không có các biện pháp điều tiết từ xa.
\end{itemize}

\section{HƯỚNG PHÁT TRIỂN}

Dựa trên các kết quả và hạn chế đã phân tích, nhóm đề xuất các hướng phát triển tiếp theo cho đề tài:

\begin{enumerate}
    \item \textbf{Phối hợp đa tác tử (Multi-Agent Coordination):} Mở rộng từ điều khiển độc lập sang phối hợp toàn mạng lưới sử dụng Multi-Agent Reinforcement Learning (MARL) hoặc Graph Neural Networks (GNN) để mô hình hóa tương tác giữa các giao lộ.
    \item \textbf{Kiểm chứng thực tế (Real-World Validation):} Triển khai thử nghiệm hệ thống trên dữ liệu giao thông thực tế (Real-world dataset) hoặc thí điểm quy mô nhỏ (Pilot test) để đánh giá độ tin cậy.
    \item \textbf{Tích hợp dữ liệu đa nguồn:} Bổ sung dữ liệu thời tiết (mưa, sương mù) và phân loại phương tiện chi tiết (xe buýt, xe ưu tiên) vào mô hình dự báo và điều khiển để tăng độ chính xác và tính thực tiễn.
    \item \textbf{Điều khiển dự báo (Predictive Control):} Phát triển khả năng dự báo dài hạn (10-15 phút) để hệ thống có thể lập kế hoạch điều tiết chủ động (Proactive) thay vì chỉ phản ứng lại tình trạng hiện tại (Reactive).
\end{enumerate}