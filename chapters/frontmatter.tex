\chapter*{Lời Cam Đoan}
Với tư cách là người thực hiện khóa luận tốt nghiệp này, chúng tôi là Ngô Trọng Nghĩa, mã số sinh viên 21161155 và Nguyễn Nhựt Đăng, mã số sinh viên 21119062 cùng đang theo học ngành Công nghệ kỹ thuật máy tính tại Khoa Điện - Điện tử, Trường Đại học Công Nghệ Kỹ Thuật TP.HCM. Chúng tôi xin khẳng định đây hoàn toàn là công trình nghiên cứu do chúng tôi sáng tạo. Nội dung và kết quả của khóa luận phản ánh năng lực chuyên môn, kỹ năng nghiên cứu và sự nỗ lực tự thân của chúng tôi, không hề vay mượn hay sao chép từ bất kỳ đồ án, bài báo hay tài liệu nào đã được công bố trước đây mà không trích dẫn nguồn. Chúng tôi cam đoan mọi tài liệu tham khảo được sử dụng đều đã được ghi nhận đầy đủ và chính xác, tuân thủ nghiêm ngặt quy định về trích dẫn của Nhà trường và các chuẩn mực học thuật quốc tế. Chúng tôi đảm bảo tính xác thực, khách quan của thông tin trình bày và khẳng định không có hành vi học thuật không trung thực. Chúng tôi hoàn toàn chịu trách nhiệm về tính nguyên bản của công trình này và chấp nhận mọi hình thức xử lý kỷ luật nếu phát hiện bất kỳ vi phạm nào đối với bản cam kết này.
\begin{table}[!h]
\centering
\begin{tabular}{p{3cm} p{3cm} p{3cm} p{3cm}}
&  & \multicolumn{2}{c}{Nhóm sinh viên thực hiện khóa luận} \\
&  & \multicolumn{2}{c}{\textit{(Ký và ghi rõ họ tên)}} \\
&  &             &            \\
&  &             &            \\
&  &             &            \\
&  &             &            \\
&  &             &            \\
&  &             &            \\
&  & \multicolumn{1}{c}{Ngô Trọng Nghĩa} & \multicolumn{1}{c}{Nguyễn Nhựt Đăng}     
\end{tabular}
\end{table}

\chapter*{Lời Cảm Tạ}

Chúng tôi nhận thức sâu sắc rằng việc hoàn thành khóa luận tốt nghiệp này không thể thực hiện được nếu thiếu đi sự đồng hành, tư vấn và hỗ trợ quý báu từ quý Thầy Cô cùng các bạn bè trong ngành Công Nghệ Kỹ Thuật Máy Tính, Khoa Điện - Điện tử, Trường Đại Học Công Nghệ Kỹ Thuật Thành phố Hồ Chí Minh. Chúng tôi xin bày tỏ lòng biết ơn chân thành nhất đến tất cả những người đã dành thời gian, công sức góp ý và giúp đỡ chúng tôi trong suốt quá trình thực hiện công trình này. Đặc biệt, sự dẫn dắt của Thầy Huỳnh Thế Thiện đóng vai trò then chốt. Những định hướng chuyên môn, lời khuyên tận tình và lộ trình nghiên cứu Thầy vạch ra ngay từ những bước đi đầu tiên đã ảnh hưởng sâu sắc đến tư duy và cách tiếp cận đề tài của chúng tôi. Dù đã rất cố gắng, khóa luận chắc chắn vẫn còn tồn tại những điểm chưa hoàn thiện. Chúng tôi rất mong nhận được những góp ý thẳng thắn và đánh giá khách quan để có thể nâng cao kiến thức, khắc phục hạn chế và tạo ra những sản phẩm chất lượng hơn trong tương lai. Xin chân thành cảm ơn!

\chapter*{Tóm Tắt}

Nghiên cứu này đề xuất một hệ thống tích hợp dựa trên học sâu và mô phỏng giao thông nhằm dự báo lưu lượng phương tiện và hỗ trợ điều chỉnh tín hiệu đèn giao thông trong môi trường đô thị. Trọng tâm của nghiên cứu là khai thác dữ liệu hình ảnh snapshot thu thập từ camera giao thông công cộng để trích xuất các thông số giao thông quan trọng, từ đó phục vụ cho mô phỏng và dự báo. Do đặc thù dữ liệu ảnh giao thông chịu ảnh hưởng mạnh bởi điều kiện ánh sáng, thời tiết và hiện tượng che khuất, nghiên cứu áp dụng các bước tiền xử lý và tăng cường chất lượng ảnh nhằm nâng cao độ tin cậy của quá trình phân tích.

Trong hệ thống đề xuất, mô hình YOLOv11n được sử dụng để thực hiện nhiệm vụ nhận diện và đếm phương tiện từ ảnh snapshot, cho phép trích xuất các đặc trưng như lưu lượng xe, mật độ giao thông và phân bố phương tiện theo thời gian. Các thông tin này sau đó được chuyển đổi thành chuỗi dữ liệu thời gian có cấu trúc và tích hợp vào mô hình mạng nơ-ron hồi tiếp LSTM để dự báo lưu lượng xe trong ngắn hạn. Trong nghiên cứu này, mô hình LSTM được sử dụng theo hai mục đích khác nhau: (i) dự báo và trực quan hóa xu hướng lưu lượng giao thông trong tương lai phục vụ phân tích và đánh giá, và (ii) khai thác thông tin lịch sử của hệ thống giao thông (lưu lượng phương tiện, mật độ giao thông và trạng thái tín hiệu đèn) để có thể đưa ra chu kỳ đèn hợp lý. Trên cơ sở các tham số điều khiển đèn giao thông được hiệu chỉnh từ kết quả dự báo, các kịch bản điều khiển tín hiệu được xây dựng và mô phỏng trên phần mềm SUMO nhằm tái hiện trạng thái giao thông tại khu vực nghiên cứu và đánh giá hiệu quả của các chiến lược điều khiển thích nghi.

Hệ thống được thiết kế theo hướng mô-đun, cho phép kết hợp hiệu quả giữa xử lý ảnh, học sâu và mô phỏng giao thông mà không yêu cầu xử lý dữ liệu thời gian thực. Kết quả thực nghiệm cho thấy phương pháp có khả năng dự báo lưu lượng xe chính xác và hỗ trợ đánh giá hiệu quả các chiến lược điều chỉnh tín hiệu giao thông. Những kết quả này cho thấy tiềm năng ứng dụng của hệ thống trong quản lý giao thông đô thị.

