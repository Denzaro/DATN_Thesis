\chapter*{Lời Cam Đoan}
Với tư cách là người thực hiện khóa luận tốt nghiệp này, chúng tôi là Ngô Trọng Nghĩa, mã số sinh viên 21161155 và Nguyễn Nhựt Đăng, mã số sinh viên 21119062 cùng đang theo học ngành Công nghệ kỹ thuật máy tính tại Khoa Điện - Điện tử, Trường Đại học Sư phạm Kỹ thuật TP.HCM. Chúng tôi xin khẳng định đây hoàn toàn là công trình nghiên cứu do chúng tôi sáng tạo. Nội dung và kết quả của khóa luận phản ánh năng lực chuyên môn, kỹ năng nghiên cứu và sự nỗ lực tự thân của chúng tôi, không hề vay mượn hay sao chép từ bất kỳ đồ án, bài báo hay tài liệu nào đã được công bố trước đây mà không trích dẫn nguồn. Chúng tôi cam đoan mọi tài liệu tham khảo được sử dụng đều đã được ghi nhận đầy đủ và chính xác, tuân thủ nghiêm ngặt quy định về trích dẫn của Nhà trường và các chuẩn mực học thuật quốc tế. Chúng tôi đảm bảo tính xác thực, khách quan của thông tin trình bày và khẳng định không có hành vi học thuật không trung thực. Chúng tôi hoàn toàn chịu trách nhiệm về tính nguyên bản của công trình này và chấp nhận mọi hình thức xử lý kỷ luật nếu phát hiện bất kỳ vi phạm nào đối với bản cam kết này.
\begin{table}[!h]
\centering
\begin{tabular}{p{3cm} p{3cm} p{3cm} p{3cm}}
&  & \multicolumn{2}{c}{Người thực hiện tiểu luận} \\
&  & \multicolumn{2}{c}{\textit{(Ký và ghi rõ họ tên)}} \\
&  &             &            \\
&  &             &            \\
&  &             &            \\
&  &             &            \\
&  &             &            \\
&  &             &            \\
&  & \multicolumn{2}{c}{Lê Trường Thịnh}     
\end{tabular}
\end{table}

\chapter*{Lời Cảm Tạ}

Chúng tôi nhận thức sâu sắc rằng việc hoàn thành khóa luận tốt nghiệp này không thể thực hiện được nếu thiếu đi sự đồng hành, tư vấn và hỗ trợ quý báu từ quý Thầy Cô cùng các bạn bè trong ngành Hệ thống nhúng và IoT, Khoa Điện - Điện tử, Trường Đại Học Sư Phạm Kỹ Thuật Thành phố Hồ Chí Minh. Chúng tôi xin bày tỏ lòng biết ơn chân thành nhất đến tất cả những người đã dành thời gian, công sức góp ý và giúp đỡ chúng tôi trong suốt quá trình thực hiện công trình này. Đặc biệt, sự dẫn dắt của Thầy Huỳnh Thế Thiện đóng vai trò then chốt. Những định hướng chuyên môn, lời khuyên tận tình và lộ trình nghiên cứu Thầy vạch ra ngay từ những bước đi đầu tiên đã ảnh hưởng sâu sắc đến tư duy và cách tiếp cận đề tài của chúng tôi. Dù đã rất cố gắng, khóa luận chắc chắn vẫn còn tồn tại những điểm chưa hoàn thiện. Chúng tôi rất mong nhận được những góp ý thẳng thắn và đánh giá khách quan để có thể nâng cao kiến thức, khắc phục hạn chế và tạo ra những sản phẩm chất lượng hơn trong tương lai. Xin chân thành cảm ơn!

\chapter*{Tóm Tắt}

Nghiên cứu này giới thiệu một phương pháp mới dựa trên mô hình học sâu để nhận diện tín hiệu 5G (fifth-generation), còn được gọi là NR (new radio), và LTE (long-term evolution), với trọng tâm là xác định các vùng phổ tần số của tín hiệu được điều chế trong mạng vô tuyến. Phương pháp này nhằm mục đích hỗ trợ việc xây dựng các mạng vô tuyến nhận thức thế hệ tiếp theo.
Về mặt lý thuyết, trong quá trình truyền dẫn, các tín hiệu được điều chế thường trở nên khó nhận diện do có dạng sóng mang phức tạp. Để giải quyết vấn đề này, các tín hiệu thu được từ máy thu sẽ được chuyển đổi thành hình ảnh phổ, giúp hiển thị thông tin trực quan hơn bằng cách áp dụng phép biến đổi Fourier thời gian ngắn (short-time Fourier transform - STFT).

Để xác định vùng phổ của tín hiệu 5G và LTE cùng tồn tại trên một hình ảnh phổ, tác giả giới thiệu một phương pháp cảm biến phổ tiên tiến dành cho các mạng không dây thế hệ tiếp theo, sử dụng kiến trúc hai đường dẫn. Mô hình đổi mới này được thiết kế để phân đoạn chính xác các tín hiệu 5G NR và LTE bằng cách xác định nội dung phổ dựa trên tần số và thời gian mà các tín hiệu chiếm dụng. Phương pháp này tích hợp một đường dẫn ngữ cảnh nhằm thu thập thông tin ngữ nghĩa ở mức độ cao, một đường dẫn không gian để bảo toàn các đặc trưng chi tiết về không gian, và một cơ chế hợp nhất đặc trưng mới nhằm kết hợp hiệu quả thông tin từ cả hai đường dẫn. Kiến trúc này có khả năng học tập cả các đặc trưng phổ cục bộ và toàn cục, qua đó nâng cao đáng kể hiệu suất phân đoạn.
Kết quả thực nghiệm cho thấy phương pháp này đạt hiệu quả và hiệu suất vượt trội, với một kiến trúc gọn nhẹ chỉ gồm 7 triệu tham số, đạt được độ chính xác toàn cục (global accuracy) là $97.25\%$ và giá trị trung bình của chỉ số giao nhau trên hợp (mean intersection over union -- IoU) là $94.76\%$. Những kết quả này chứng minh rằng đây là một giải pháp đầy hứa hẹn dành cho các hệ thống thông tin không dây thế hệ tiếp theo.

